\documentclass{book}
\usepackage{xcolor}
\usepackage{tcolorbox}
\usepackage{amsthm, amsfonts, amssymb, amsmath, fullpage}
\usepackage{graphicx}
\usepackage{tikz}
\usepackage{hyperref}
\usepackage{enumerate}
\tcbuselibrary{theorems}
\usepackage{bbm}
\usepackage[toc,page]{appendix}

% TikZ & Figure
\usetikzlibrary{positioning}
%\usepackage{chngcntr}
%\counterwithin{figure}{section}

%Fonts
%\usepackage{mathpazo}
%\usepackage{kpfonts}
%\usepackage{libertine}
%\usepackage{XCharter}
\usepackage{palatino, newpxmath}
%\usepackage{euler}

%Useless packages
\usepackage{lipsum}

%defined commands
%%%%%%%%%%%%%
% theorem, corollary, lemma, proposition, claim
\newtcbtheorem[number within=section]{theorem}{Theorem}{colback=blue!10!white, fonttitle=\bfseries}{thm}
\newtcbtheorem[number within=section]{lemma}{Lemma}{colback=green!5!white, fonttitle=\bfseries}{lem}
\newtcbtheorem[number within=section]{proposition}{Proposition}{colback=pink!10!white}{prop}
\newtcbtheorem[use counter from=theorem]{corollary}{Corollary}{colback=blue!5!white}{cor}

% definition, examples, problem, solution, remark
\newtcbtheorem[number within=section]{definition}{Definition}{colback=cyan!5!white, fonttitle=\bfseries}{def}
\newtcbtheorem[number within=section]{problem}{Problem}{colback=white}{prob}
\newtheorem{example}{Example}[section]
\newtheorem*{solution}{Solution}
\newtheorem*{remark}{Remark}
\newtheorem{claim}{Claim}[section]

%defined symbols
\newcommand{\cC}{\mathcal{C}}
\newcommand{\cF}{\mathcal{F}}
\newcommand{\cR}{\mathcal{R}}
\newcommand{\R}{\mathbb{R}}
\newcommand{\N}{\mathbb{N}}
\newcommand{\Z}{\mathbb{Z}}
